A indústria de transporte desempenha um papel vital na economia global, garantindo o fluxo de mercadorias e serviços. As transportadoras enfrentam desafios operacionais, logísticos e econômicos complexos, que exigem abordagens inovadoras para aprimorar suas atividades e se manterem competitivas em um mercado em constante evolução. A Engenharia de Produção, com sua abordagem sistêmica e conhecimentos multidisciplinares em Pesquisa Operacional, Logística, Economia e Computação, destaca-se como uma disciplina capaz de enfrentar esses desafios por meio da construção de um sistema de informações.

Um sistema de informações bem projetado e implementado desempenha um papel crucial na resolução de problemas em uma transportadora. Ele consiste em uma infraestrutura tecnológica e processos organizacionais para coletar, armazenar, processar e disseminar informações relevantes em tempo real. Para isso será necessária a coleta sistemática de dados precisos sobre todas as atividades da transportadora, como roteirização, tempos de entrega, desempenho de motoristas e custos operacionais. Esses dados fornecem uma visão completa das operações, identificando gargalos, pontos de ineficiência e áreas que requerem melhorias.