A indústria de transporte desempenha um papel vital na economia global, sendo a espinha dorsal que garante o fluxo contínuo de mercadorias e serviços em todo o mundo. No entanto, essa indústria enfrenta desafios complexos e multifacetados que afetam sua eficiência e competitividade. Problemas relacionados às operações logísticas, como o planejamento de rotas, a gestão de estoques e a coordenação de entregas, tornam-se ainda mais desafiadores em um cenário de constante evolução tecnológica e exigências crescentes dos clientes.

Além disso, questões econômicas também desempenham um papel crítico. As transportadoras enfrentam pressões para reduzir custos operacionais, ao mesmo tempo em que precisam oferecer serviços de qualidade e atender às demandas específicas dos clientes. A falta de controle adequado dos custos e receitas pode levar a margens de lucro apertadas e afetar negativamente a saúde financeira da empresa.

Nesse contexto, o sistema de informações proposto, baseado na abordagem sistêmica e conhecimentos multidisciplinares da Engenharia de Produção, surge como uma solução estratégica e inovadora para enfrentar esses desafios. Através da integração de Pesquisa Operacional, Logística, Economia e Computação, o sistema busca criar uma plataforma que permita à transportadora coletar, armazenar, processar e disseminar informações relevantes em tempo real.

A compreensão dos processos existentes é fundamental para identificar os gargalos e ineficiências no fluxo de trabalho. A coleta e análise de dados precisos sobre todas as atividades da transportadora, como roteirização, tempos de entrega, desempenho de motoristas e custos operacionais, fornecem uma visão completa das operações. Isso possibilita a identificação de áreas que requerem melhorias, a otimização das rotas de transporte e o controle eficiente dos estoques.

Um dos pontos problemáticos é a falta de informações centralizadas e integradas. Com múltiplos sistemas e fontes de dados, a tomada de decisões pode ser dificultada e a coordenação entre departamentos pode ser prejudicada. A ausência de uma visão holística das operações pode resultar em decisões subótimas e redundâncias no processo.

Outro desafio é a necessidade de melhorar a satisfação do cliente e fornecer uma experiência positiva ao longo de toda a jornada. Acompanhar o status das entregas, fornecer informações em tempo real e permitir que os clientes avaliem o serviço são aspectos cruciais para construir confiança e fidelidade com a base de clientes.

Diante desses desafios, a solução proposta tem como objetivo aumentar a eficiência operacional da transportadora, aprimorar a gestão de estoques, centralizar e integrar informações, otimizar a roteirização e aumentar a satisfação do cliente. O sistema de informações permitirá o compartilhamento transparente de dados entre os diferentes atores envolvidos, possibilitando a colaboração e a coordenação eficiente das atividades.

Ao implementar esse sistema, a transportadora terá uma visão mais precisa e abrangente de suas operações, possibilitando a identificação de oportunidades de melhoria e a adoção de estratégias mais eficazes. Isso resultará em processos mais eficientes, entregas mais rápidas, redução de custos operacionais e, ao mesmo tempo, o aumento da satisfação dos clientes.
