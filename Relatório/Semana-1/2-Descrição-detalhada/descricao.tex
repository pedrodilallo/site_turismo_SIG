A indústria do turismo desempenha um papel significativo na economia global, proporcionando experiências únicas e enriquecedoras aos viajantes. No entanto, essa indústria também enfrenta desafios complexos relacionados ao planejamento de roteiros turísticos, que afetam diretamente a satisfação dos clientes e a eficiência operacional das empresas. Problemas como a falta de otimização das rotas, a falta de informações centralizadas e a ausência de uma visão holística das operações podem levar a itinerários ineficientes, insatisfação dos turistas e, por consequência, impactar negativamente a reputação das empresas do setor.

Nesse contexto, o sistema de informação proposto surge como uma solução estratégica e inovadora para enfrentar os desafios específicos do planejamento de roteiros turísticos. A abordagem busca integrar de forma sinérgica conhecimentos e técnicas que permitam às empresas do setor coletar, processar e utilizar informações relevantes para a criação de itinerários altamente satisfatórios e eficientes.

A compreensão das necessidades e preferências dos clientes é fundamental para criar experiências turísticas sob medida. A coleta de dados detalhados sobre as atrações, horários de funcionamento, tempo de visita e níveis de satisfação dos turistas fornece uma base sólida para o planejamento de roteiros otimizados. A utilização de modelos de Pesquisa Operacional, como o Problema de Orientação com Janelas de Tempo (OPTW) ou o Problema de Caixeiro Viajante Lucrativo com Restrições Temporais e Orçamentárias, permitirá a criação de algoritmos eficazes para resolver esses problemas complexos, garantindo a sequência lógica e eficiente das atividades turísticas.

Um dos principais desafios é a centralização e integração das informações. Com a utilização de um sistema de informações, as empresas poderão ter uma visão holística das operações, permitindo uma coordenação eficiente entre os diversos atores envolvidos, como hotéis, restaurantes e atrações turísticas. Isso possibilitará a sincronização das atividades e a criação de itinerários personalizados, considerando as restrições temporais e orçamentárias dos clientes, ao mesmo tempo em que maximiza a satisfação com a experiência turística.

Outro aspecto crucial é a necessidade de melhorar a satisfação do cliente ao longo de toda a jornada. O sistema de informação permitirá o monitoramento em tempo real das atividades dos turistas, fornecendo informações atualizadas sobre as atrações visitadas, horários disponíveis, pausas para refeições e intervalos para descanso. Além disso, a coleta de feedback dos clientes após cada visita permitirá avaliar a qualidade dos serviços prestados e realizar ajustes para aprimorar continuamente a experiência do turista.

Com isso, podemos estabelecer, já, os atores envolvidos no sistema. São eles:

\subsection{Clientes:} 
Os clientes são os principais usuários do sistema. Eles interagem com o sistema para acessar informações sobre atrações turísticas, horários de funcionamento, custos de visitação e outros detalhes relevantes para planejar suas viagens. Além disso, os clientes fornecem feedback sobre suas experiências, o que é essencial para aprimorar o sistema e garantir a satisfação do cliente a longo prazo. O intuito é que os clientes forneçam apenas as cidades que querem visitar e recebam a rota otimizada para aquela localidade. Os clientes também poderão consultar rotas que já foram criadas pelo sistema.

\subsection{API do Google sobre as cidades:} 
A API do Google sobre as cidades é um agente externo ao sistema, que fornece informações valiosas sobre as cidades e atrações turísticas. Essa API pode incluir dados como pontos turísticos, hotéis, restaurantes, horários de funcionamento, informações geográficas e muito mais. A integração dessa API no sistema permite que os usuários acessem informações atualizadas e detalhadas sobre os destinos turísticos, contribuindo para a precisão e relevância dos roteiros gerados.

\subsection{Sistema de Otimização:} 
O sistema de otimização é o núcleo do sistema de informação. Ele utiliza técnicas de Pesquisa Operacional para criar itinerários turísticos eficientes e satisfatórios. O sistema de otimização analisa os dados fornecidos pela API do Google sobre as cidades, juntamente com as preferências e restrições dos clientes, para criar roteiros personalizados que atendam às necessidades e desejos dos viajantes.

\subsection{Sistema de Coleta de dados:} 
Este é o responsável por consultar a API e, além disso, fazer toda a gestão dos dados do sistema. Isso inclui a formatação das informações fornecidas pela API, os dados dos clientes e o registro das rotas que já foram otimizadas.

\subsection{Administrador do Sistema:} 
O administrador do sistema é um agente humano responsável por gerenciar e supervisionar o funcionamento do sistema como um todo. Ele atua como um ponto central de controle, garantindo que o sistema esteja atualizado, seguro e funcionando adequadamente. O administrador é responsável por lidar com a manutenção do sistema, realizar atualizações, monitorar a segurança dos dados e fornecer suporte técnico aos usuários. Além disso, o administrador também pode realizar análises dos dados gerados pelo sistema para identificar oportunidades de melhoria e tomar decisões estratégicas para aprimorar a eficiência e a qualidade dos serviços oferecidos pelo sistema.

\subsection{Sistema de Feedback:}

O sistema de feedback desempenha um papel crucial no processo de comunicação entre os clientes e o administrador do sistema. Ele atua como um agente intermediário que coleta e organiza os feedbacks dos clientes e armazena dados sobre as rotas e sua eficência, fornecendo informações valiosas para o administrador tomar decisões informadas e realizar melhorias contínuas nos serviços oferecidos.