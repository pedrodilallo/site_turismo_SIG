\subsection{Análise de Requisitos}


\subsubsection{Cadastro de Novos Clientes}
O sistema deve proporcionar uma interface de cadastro de novos clientes, onde os usuários possam preencher informações pessoais, como nome, endereço de e-mail e preferências de viagem. Essas preferências incluem as cidades ou atrações turísticas que eles desejam visitar durante sua viagem. O cadastro é essencial para personalizar os roteiros de acordo com as escolhas individuais de cada cliente, oferecendo uma experiência turística mais adequada às suas expectativas.

\subsubsection{Receber Cidades ou Atrações de Interesse dos Clientes}
Após o cadastro, o sistema permite que os clientes selecionem as cidades ou atrações turísticas que mais lhes interessam. Eles podem indicar os pontos turísticos específicos que desejam visitar, bem como as atividades ou eventos que gostariam de incluir em seus roteiros. Essa funcionalidade permite que os clientes expressem suas preferências de forma clara, tornando possível a criação de roteiros personalizados e alinhados com seus interesses individuais.

\subsubsection{Acessar a API do Google para Coletar Dados sobre as Cidades}
Para fornecer informações detalhadas sobre as cidades e atrações turísticas selecionadas pelos clientes, o sistema integra-se à API do Google ou a outras fontes confiáveis de dados. Por meio dessa API, o sistema pode obter dados relevantes, como horários de funcionamento, avaliações, endereços e descrições detalhadas dos pontos turísticos. A coleta dessas informações é fundamental para a criação de roteiros precisos e bem informados.

\subsubsection{Formatar e Computar os Dados Adquiridos}
Os dados obtidos da API do Google precisam ser tratados e formatados adequadamente para que possam ser usados no processo de otimização de rotas. As informações coletadas devem ser armazenadas em uma estrutura de dados apropriada, facilitando sua manipulação e utilização pelo sistema durante o processo de criação de roteiros turísticos otimizados.

\subsubsection{Otimizar Rotas}
Uma das principais funcionalidades do sistema é a otimização das rotas turísticas. Com base nas cidades e atrações selecionadas pelos clientes, o sistema deve criar roteiros que maximizem a satisfação do cliente enquanto minimizam custos e tempo de deslocamento. Durante a otimização, o sistema considera restrições de tempo, horários de funcionamento das atrações e possíveis custos associados a cada visita.

\subsubsection{Consultar Relatórios sobre as Rotas}
O sistema deve gerar relatórios detalhados sobre as rotas otimizadas. Esses relatórios fornecem informações como a sequência de visitação das atrações, o tempo estimado de permanência em cada local, distâncias percorridas, custos totais e outras métricas relevantes. Os relatórios são úteis tanto para o cliente, que pode acompanhar seu itinerário, quanto para o administrador do sistema, que pode analisar o desempenho das rotas e realizar melhorias contínuas.

\subsubsection{Executar Pedidos de Otimização}
O sistema permite que os clientes solicitem a otimização de suas rotas mesmo após a conclusão do roteiro inicial. Caso haja mudanças de planos ou novos interesses que surjam durante a viagem, o cliente pode solicitar uma nova otimização para seu roteiro. Essa flexibilidade garante que o cliente tenha a melhor experiência possível, adaptando seu itinerário de acordo com suas necessidades e desejos.

\subsubsection{Verificar Satisfação das Restrições}
Durante o processo de otimização de rotas, o sistema verifica se as restrições impostas pelo cliente, como horários de funcionamento das atrações e intervalos de tempo disponíveis para visitação, são atendidas pelo roteiro otimizado. Caso alguma restrição não seja satisfeita, o sistema oferece alternativas ou ajusta o roteiro de forma adequada, assegurando que o cliente possa aproveitar ao máximo suas visitas e experiências turísticas.

\subsubsection{Registro de Rotas Já Otimizadas em uma Base de Dados}
O sistema registra as rotas já otimizadas em uma base de dados para fins de histórico e análise. Esses registros permitem que o administrador e os clientes possam acessar roteiros previamente otimizados, evitando retrabalhos e facilitando a consulta de itinerários anteriores. Essa funcionalidade também permite que o sistema aprenda com experiências anteriores e refine suas estratégias de otimização com base em dados históricos, melhorando continuamente o serviço oferecido.

\subsection{Diagrama de Caso de Uso}
