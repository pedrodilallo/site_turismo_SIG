Em conclusão, o projeto realizado apresentou uma abordagem inovadora e promissora para otimização de roteiros turísticos através da integração de um sistema de informação gerencial com um modelo baseado na metaheurística de busca tabu. O modelo em questão consiste no estabelecimento da rota de maior satisfação ao cliente levando em conta restrições temporais, espaciais e orçamentárias.

A sinergia entre os dois modelos mostrou-se altamente eficiente na busca pela melhor combinação de pontos turísticos, considerando fatores como tempo, custo e distâncias percorridas. O sistema proposto não apenas proporciona uma experiência mais agradável e satisfatória aos viajantes, mas também contribui para o desenvolvimento sustentável do setor turístico, ao otimizar o fluxo de visitantes e minimizar impactos negativos ao meio ambiente.

É importante destacar que, apesar dos avanços alcançados, existem ainda desafios a serem superados, como o aprimoramento dos algoritmos utilizados, a inclusão de novos critérios de otimização e a adaptação contínua às mudanças nos padrões de comportamento dos turistas. Além disso, a implementação bem-sucedida do modelo requer uma colaboração estreita entre profissionais do turismo e especialistas em tecnologia da informação.

O presente projeto segue em desenvolvimento, buscando melhores e mais sofisticadas integrações com API's e corrigindo falhas dos modelos anteriores. A equipe está comprometida em aprimorar continuamente o sistema e expandir sua funcionalidade, visando oferecer aos usuários uma ferramenta cada vez mais completa e eficiente para o planejamento de suas viagens. Esse desenvolvimento pode ser acompanhado no GitGub, plataforma para armazenar repositórios git na nuvem. O link para o repositório pode ser encontrado em: \url{https://github.com/pedrodilallo/site_turismo_SIG}.

Para acessar um repositório no GitHub, é necessário seguir alguns passos simples. Primeiramente, é preciso criar uma conta no GitHub, caso ainda não possua uma. Depois de fazer o login, navegue até a página inicial do repositório que deseja acessar. No canto superior direito, você encontrará um botão verde intitulado "Code". Clique nele para abrir o menu suspenso. A partir daí, você pode optar por clonar o repositório usando o Git em seu terminal, ou pode baixá-lo como um arquivo ZIP diretamente para o seu computador. Caso decida clonar o repositório, copie a URL fornecida e utilize o comando "git clone" seguido da URL em seu terminal. Dessa forma, você terá acesso aos arquivos e histórico de commits do repositório, permitindo colaborar, contribuir e acompanhar o desenvolvimento do projeto.