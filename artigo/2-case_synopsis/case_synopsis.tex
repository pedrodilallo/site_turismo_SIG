In the realm of modern travel, planning a trip and its subsequent route presents a multifaceted challenge. The digital age, while offering a plethora of information, often inundates travelers with an overwhelming array of choices, from destinations to accommodations. Balancing such vast options with budgetary and time constraints further complicates the endeavor. Additionally, the intricacies of route planning, which demand efficiency and adaptability, especially in unfamiliar terrains, add another layer of complexity. These challenges, coupled with cultural, linguistic, and logistical considerations, underscore the intricate nature of travel planning in contemporary times. As such, meticulous research, aided by technological advancements, remains paramount in navigating this intricate landscape.

\subsection*{System Requirements}

\subsubsection*{Requiriment 1 - Registering Customers:} The system should be able to register new customers. They should provide personal information, such as name, e-mail, and address. This functionality is essential to identify and analyse data from system users.

\subsubsection*{Requirement 2 - Selecting Cities or tourist Attractions:} Custumers must be able to select the cities or attractions upon which their planned route will be based on.  This functionality aims for itinerary personalization according to customer prefference.

\subsubsection*{Requirement 3 - Colecting City Data:} The system will have to mine data from the internet to obtain relevant information about the cities, attractions, hotels and restaurants selected by the customer. This information include opening and closing time windows, cost, geolocation and all parameters relevant to the optimization model. 

\subsubsection*{Requirement 4 - Formmating Acquired Data:} After the data colletion, the system must format it to be able to run the model and enable route optimization.

\subsubsection*{Requirement 5 - Route Optimizations:} This is the core functionality of the system. It must be able to plan a good route based on the data acquired from the internet. The optimization will have to take into account customers prefferences and maximize their satisfaction.

\subsubsection*{Requirement 6 - Create reports with route data} The Optimization System should be able to produce detailed reports on optimized tourist routes. These reports provide valuable information, such as generation dates, who made the solicitation, most selected cities, and more, on the generated itineraries, allowing for more informed analyses and decision-making.

\subsubsection*{Requirement 7 - Collect Feedback:} Customers should be able to send feedback about tourist itineraries and their overall experience with the system. This functionality allows users to express their opinions and suggestions for the system's continuous improvement.


\subsubsection*{Requirement 11 - Store Customer Data and Feedback:} The Feedback System should be able to store registered customer data and the feedback they may send. This feature is crucial to ensure data persistence and allow for future analyses.

\subsubsection*{Requirement 12 - Create reports with Feedback data:} The Feedback System should produce detailed reports on the feedback sent by clients. These reports are valuable for analyzing the suggestions and experiences reported by users.

\subsection*{Use Case Diagram}

